\chapter{Технологическая часть}

\section{Выбор операционной системы}

Согласно требованиям технического задания, разрабатываемый портал должен обладать высокой доступностью, работать на типичных архитектурах ЭВМ (Intel x86, Intel x64), а так же быть экономически недорогим для сопровождения. Таким образом, требования к ОС следующие.
\begin{itemize}
	\item \textbf{Распространенность}. На рынке труда должно быть много специалистов, способных администрировать распределенную систему, работающую под управлением выбранной операционной системы.
	\item \textbf{Надежность}. Операционная система должна широко использоваться в стабильных проектах, таких как Mail.Ru, Vk.com, Google.com. Эти компании обеспечивают высокую работоспособность своих сервисов, и на их опыт можно положиться.
	\item \textbf{Наличие требуемого программного обеспечения}. Выбор операционной системы не должен ограничивать разработчиков в выборе программного обеспечения, библиотек.
	\item \textbf{Цена}.
\end{itemize}
 
Под данные требования лучше всего подходит ОС Linux, дистрибутив Ubuntu. \textbf{Ubuntu} \cite{ubuntu} — это дистрибутив, использующий ядро Linux. Как и все дистрибутивы Linux, Ubuntu является ОС с открытым исходным кодом, бесплатным для использования. Поставляется как в клиентской (с графическим интерфейсом), так и в серверной (без графического интерфейса) версиями. Ubuntu поставляется с современными версиями ПО. Преимуществом Ubuntu являются низкие требования к квалификации системных администраторов. Однако Ubuntu менее стабильна в работе.


\section{Выбор СУБД}

В качестве СУБД была выбрана \textbf{PostgreSQL} \cite{postgresql}, так как она наилучшим
образом подходит под требования разрабатываемой системы:
\begin{itemize}
	\item Масштабируемость: PostgreSQL поддерживает горизонтальное масштабирование, что позволяет распределить данные и запросы между несколькими узлами базы данных. Это особенно полезно в географически распределенных системах, где данные и пользователи могут быть разбросаны по разным регионам.
	\item Географическая репликация: PostgreSQL предоставляет возможность
настройки репликации данных между различными узлами базы данных, расположенными в разных географических зонах. Это позволяет
обеспечить отказоустойчивость и более быстрый доступ к данным для
пользователей из разных частей нашей страны.
	\item Гибкость и функциональность: PostgreSQL обладает широким набором
функций и возможностей, что делает его подходящим для различных
типов приложений и использования в распределенной среде. Он поддерживает сложные запросы, транзакции, хранимые процедуры и многое
другое.
	\item Надежность и отказоустойчивость: PostgreSQL известен своей надежностью и стабильностью работы. В распределенной географической
системе это особенно важно, поскольку он способен обеспечить сохранность данных и доступность даже при сбоях в отдельных узлах.
\end{itemize}


\textbf{Выбор языка разработки и фреймворков компонент
портала}

Для разработки бэкенда существует множество языков программирования, каждый из которых имеет свои сильные и слабые стороны. Рассмотрим наиболее популярные языки: Python, Java, JavaScript (Node.js), Go и Ruby:

\begin{enumerate}
	\item Python.
	\begin{enumerate}
		\item Плюсы.
		\begin{enumerate}
			\item Простота и читаемость кода: Python славится своим чистым синтаксисом, что делает его одним из самых легких языков для изучения и поддержки. Это особенно важно для командной работы и проектов, которые должны быть легко расширяемыми.
			\item Богатая экосистема: Существуют мощные фреймворки, такие как Django и Flask, которые значительно ускоряют разработку веб-приложений. Python также имеет огромное количество библиотек для работы с базами данных, кешированием, очередями и другими бэкенд-задачами.
			\item Гибкость: Python применим не только для веб-разработки, но и для множества других задач, таких как обработка данных, машинное обучение, автоматизация и многое другое. Это делает его универсальным выбором.
			\item Сообщество и поддержка: Огромное сообщество разработчиков и обилие документации обеспечивают быстрое решение проблем и поддержку развития проектов.
		\end{enumerate}

		\item Минусы.
		\begin{enumerate}
			\item Скорость выполнения: Python является интерпретируемым языком, что делает его медленнее по сравнению с компилируемыми языками, такими как Java или Go. Однако эта проблема часто нивелируется мощностью серверов и оптимизацией кода.
		\end{enumerate}
	\end{enumerate}

	\item Java.
	\begin{enumerate}
		\item Плюсы.
		\begin{enumerate}
			\item Высокая производительность: Java компилируется в байт-код, который затем исполняется виртуальной машиной (JVM), что обеспечивает высокую производительность.
			\item Масштабируемость: Java часто используется в крупных корпоративных приложениях, благодаря своей способности легко масштабироваться.
			\item Безопасность и стабильность: Java известна своей стабильностью, что делает её подходящей для критически важных систем, требующих высокой надёжности.
		\end{enumerate}

		\item Минусы.
		\begin{enumerate}
			\item Сложность кода: Java имеет более громоздкий синтаксис по сравнению с Python, что делает разработку медленнее, а код — труднее читаемым.
			\item Более медленная разработка: Разработка на Java требует больше времени из-за необходимости писать больше кода для реализации аналогичных функций.
		\end{enumerate}
	\end{enumerate}

	\item JavaScript (Node.js).
	\begin{enumerate}
		\item Плюсы.
		\begin{enumerate}
			\item Единый язык для фронтенда и бэкенда: Если ваш проект использует JavaScript на фронтенде, то использование Node.js позволяет унифицировать стек технологий.
			\item Асинхронная обработка: Node.js работает на основе событийной модели, что позволяет эффективно управлять большим количеством запросов одновременно.
			\item Широкое сообщество: JavaScript — один из самых популярных языков, что гарантирует обилие ресурсов, библиотек и инструментов.
		\end{enumerate}

		\item Минусы.
		\begin{enumerate}
			\item Однопоточная модель: Несмотря на асинхронную природу Node.js, она может быть ограничена при выполнении тяжёлых задач, которые требуют много вычислительных ресурсов.
			\item Меньшая стабильность: JavaScript часто обновляется, а экосистема модулей может быть нестабильной по сравнению с более зрелыми языками.
		\end{enumerate}
	\end{enumerate}

	\item Go (Golang).
	\begin{enumerate}
		\item Плюсы.
		\begin{enumerate}
			\item Высокая производительность: Go компилируется в машинный код, что делает его крайне быстрым.
			\item Простота в управлении многозадачностью: Go предлагает встроенную поддержку параллелизма через горутины, что облегчает создание масштабируемых приложений.
			\item Меньше накладных расходов: Go подходит для создания микросервисов и работы с высоконагруженными системами.
		\end{enumerate}

		\item Минусы.
		\begin{enumerate}
			\item Ограниченная экосистема: Хотя Go активно развивается, экосистема библиотек и фреймворков не так богата, как у Python или Java.
			\item Меньшая гибкость: Go ориентирован на высокую производительность, но его синтаксис менее гибок для других задач, таких как анализ данных или машинное обучение.
		\end{enumerate}
	\end{enumerate}

	\item Ruby.
	\begin{enumerate}
		\item Плюсы.
		\begin{enumerate}
			\item Простота и элегантность: Ruby, как и Python, предлагает удобный и понятный синтаксис, что ускоряет разработку.
			\item Фреймворк Ruby on Rails: Один из самых популярных фреймворков для веб-разработки, который значительно ускоряет создание веб-приложений.
		\end{enumerate}

		\item Минусы.
		\begin{enumerate}
			\item Производительность: Ruby медленнее, чем многие другие языки (например, Go или Java), что делает его менее подходящим для высоконагруженных приложений.
			\item Меньшая популярность: Несмотря на сильные стороны Ruby on Rails, язык постепенно теряет популярность, уступая Python и JavaScript.
		\end{enumerate}
	\end{enumerate}
\end{enumerate}


Таким образом, \textbf{Python} \cite{python} —- это оптимальный выбор для бэкенд-разработки по ряду причин:

\begin{enumerate}
	\item Простота и скорость разработки: Python позволяет разработчикам писать меньше кода и быстрее реализовывать функциональность, благодаря удобному синтаксису и мощным фреймворкам (Django, Flask, FastAPI). Это особенно важно для стартапов и небольших проектов, где критично быстрое создание прототипов и внедрение изменений.
	\item Широкая экосистема: В Python существует множество готовых библиотек для работы с базами данных, аутентификацией, кэшированием, очередями, обработкой данных и другими задачами бэкенда. Это ускоряет разработку и снижает количество ручной работы.
	\item Поддержка современных технологий: Python активно используется для задач, связанных с машинным обучением, обработкой данных и научными вычислениями. Это делает его выбором номер один для компаний, работающих с большими данными или развивающих искусственный интеллект.
	\item Сообщество и поддержка: Python имеет одно из крупнейших сообществ разработчиков, что упрощает решение проблем, обновление знаний и получение поддержки.
	\item Универсальность: Python может использоваться не только для разработки веб-приложений, но и для автоматизации, обработки данных и других областей, что делает его многофункциональным инструментом.
\end{enumerate}

Также выберем фреймворк для бекенд разработки. FastAPI, Django и Flask — популярные фреймворки для создания веб-приложений на Python. Каждый из них имеет свои особенности, подходы и сферы применения. Рассмотрим ключевые различия.

\begin{enumerate}
	\item FastAPI.
	\begin{enumerate}
		\item Плюсы.
		\begin{enumerate}
			\item Асинхронная природа обеспечивает высокую производительность для обработки большого числа запросов.
			\item Простота и понятность благодаря аннотациям типов и автоматической валидации данных.
			\item Интеграция с современными стандартами API и поддержка OpenAPI.
			\item Подходит для микросервисной архитектуры.
		\end{enumerate}

		\item Минусы.
		\begin{enumerate}
			\item Не такой полный фреймворк, как Django: нет встроенной поддержки ORM или административной панели "из коробки" (но это компенсируется сторонними библиотеками).
		\end{enumerate}
	\end{enumerate}

	\item Django
	\begin{enumerate}
		\item Плюсы.
		\begin{enumerate}
			\item Полноценный фреймворк для создания полнофункциональных приложений.
			\item Быстрый старт для создания сложных веб-приложений благодаря встроенным инструментам.
			\item Отличная документация и большое сообщество.
			\item Интегрированная административная панель для удобного управления данными.
		\end{enumerate}

		\item Минусы.
		\begin{enumerate}
			\item Меньшая гибкость по сравнению с более лёгкими фреймворками, такими как Flask или FastAPI.
			\item Не поддерживает асинхронность "из коробки" (сравнительно новые версии поддерживают её частично).
			\item Более тяжеловесен для небольших проектов или микросервисов, где не нужны все встроенные возможности.
		\end{enumerate}
	\end{enumerate}

	\item Flask
	\begin{enumerate}
		\item Плюсы.
		\begin{enumerate}
			\item Минимализм и простота. Легко настраивается для нужд любого проекта.
			\item Высокая гибкость и возможность выбора инструментов.
			\item Хорошо подходит для микросервисной архитектуры.
		\end{enumerate}

		\item Минусы.
		\begin{enumerate}
			\item Отсутствие встроенных решений: многие базовые вещи, такие как ORM, маршрутизация или системы шаблонов, необходимо подключать через сторонние библиотеки.
			\item Сложнее масштабировать и поддерживать крупные проекты, так как проектная структура зависит от разработчиков.
		\end{enumerate}
	\end{enumerate}
\end{enumerate}


Таким образом, для разработки был выбран фреймфорк \textbf{FastAPI} \cite{fastapi} по следующим причинам.

\begin{enumerate}
	\item Производительность: FastAPI предлагает одну из самых высоких производительностей среди Python-фреймворков. Это особенно важно для приложений, требующих быстрого отклика при большом количестве запросов, таких как API и микросервисы.
	\item Поддержка асинхронности: В отличие от Django, который частично поддерживает асинхронные операции, FastAPI изначально построен с учётом асинхронного программирования. Это делает его идеальным для приложений, где необходимо эффективно обрабатывать большое количество параллельных запросов (например, в реальном времени или при работе с внешними API).
	\item Простота и автоматическая валидация: FastAPI использует аннотации типов Python для автоматической валидации данных и генерации документации, что значительно упрощает работу с API. Это улучшает качество кода и сокращает количество ошибок.
	\item Генерация документации "из коробки": FastAPI автоматически создаёт документацию API с использованием стандартов OpenAPI и Swagger. Это экономит время разработчиков и упрощает работу с клиентами и другими командами.
	\item Гибкость и современность: FastAPI предлагает более гибкий и лёгкий подход к разработке по сравнению с Django, сохраняя простоту и читабельность кода, как в Flask. Он идеально подходит для создания быстрых, масштабируемых и лёгких приложений, особенно в микросервисной архитектуре.
\end{enumerate}


\section{Выбор фреймворка фронтенд разработки}

Фреймворки для разработки фронтенда, такие как React, Angular и Vue, предоставляют различные подходы к созданию современных веб-приложений. Рассмотрим основные отличия этих фреймворков.

\begin{enumerate}
	\item React (Библиотека для построения пользовательских интерфейсов (UI), с акцентом на компонентный подход)
	\begin{enumerate}
		\item Плюсы.
		\begin{enumerate}
			\item Широкая поддержка и большое сообщество разработчиков.
			\item Гибкость в выборе технологий для архитектуры приложения.
			\item Легко интегрируется в существующие проекты.
			\item Широко используется для мобильной разработки с помощью React Native.
		\end{enumerate}

		\item Минусы.
		\begin{enumerate}
			\item Требует настройки экосистемы (например, выбор между Redux, MobX или другими для управления состоянием).
			\item Не является полным фреймворком, что может потребовать больше усилий для конфигурации.
		\end{enumerate}
	\end{enumerate}

	\item Angular (Полноценный фреймворк, предлагающий строгую структуру для разработки приложений с двусторонней привязкой данных)
	\begin{enumerate}
		\item Плюсы.
		\begin{enumerate}
			\item Полноценный фреймворк с решениями "из коробки" для большинства задач.
			\item Поддержка TypeScript, что улучшает качество кода и упрощает работу в крупных проектах.
			\item Отличная документация и регулярные обновления от Google.
		\end{enumerate}

		\item Минусы.
		\begin{enumerate}
			\item Большая кривизна обучения по сравнению с React или Vue.
			\item Больше шаблонного кода и сложностей для выполнения простых задач.
			\item Более "тяжеловесный" по сравнению с другими решениями, что может сказываться на производительности.
		\end{enumerate}
	\end{enumerate}

	\item Vue.js (Прогрессивный фреймворк, который можно постепенно интегрировать в существующий проект)
	\begin{enumerate}
		\item Плюсы.
		\begin{enumerate}
			\item Простота использования и лёгкая кривая обучения.
			\item Высокая производительность благодаря лёгкости фреймворка.
			\item Прогрессивность: можно добавлять Vue к уже существующим проектам по мере необходимости.
		\end{enumerate}

		\item Минусы.
		\begin{enumerate}
			\item Меньшая популярность по сравнению с React и Angular, что приводит к менее богатой экосистеме и меньшему количеству библиотек.
			\item Меньшее количество крупных компаний используют Vue, что может сказаться на количестве вакансий и поддержке.
		\end{enumerate}
	\end{enumerate}
\end{enumerate}

Таким образом, был выбран фреймворк \textbf{React} \cite{react} по следующим причинам.

\begin{enumerate}
	\item Популярность и поддержка сообщества: React является одним из самых популярных инструментов для фронтенд-разработки. Благодаря этому для React доступно множество библиотек, инструментов и ресурсов. Это значительно упрощает разработку, особенно при необходимости интеграции с другими технологиями.
	\item Гибкость: В отличие от Angular, который является "жёстким" фреймворком с предустановленными решениями, React предлагает больше свободы. Разработчики могут выбирать любые библиотеки для маршрутизации, управления состоянием и работы с сервером, адаптируя проект под конкретные требования.
	\item Производительность через Virtual DOM: React использует Virtual DOM для минимизации реальных изменений в DOM, что делает его быстрым даже для сложных пользовательских интерфейсов. Это особенно важно для крупных приложений с динамически изменяющимися данными.
	\item Поддержка мобильной разработки: React Native предоставляет возможность использовать знания и компоненты React для разработки мобильных приложений под iOS и Android. Это позволяет создавать кроссплатформенные приложения с минимальными усилиями.
\end{enumerate}

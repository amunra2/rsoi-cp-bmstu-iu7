\chapter{Задание}

\section{Условие}

\textbf{Вариант 4}: Напишите программу, которая в качестве входа принимает произвольное регулярное выражение, и выполняет следующие преобразования:
\begin{enumerate}
    \item По регулярному выражению строит НКА.
    \item По НКА строит эквивалентный ему ДКА.
    \item По ДКА строит эквивалентный ему КА, имеющий наименьшее возможное количество состояний (Алгоритм Бржозовского).
    \item Моделирует минимальный КА для входной цепочки из терминалов исходной грамматики.
\end{enumerate}


\chapter{Выполнение лабораторной работы}

\section{Тесты}

\begin{table}[H]
    \centering
	\caption{Набор тестов и ожидаемые результаты работы программы}
    \label{tbl:tests}
	\begin{tabular}{|c|c|c|c|}
        \hline
        \begin{minipage}[t]{4cm}\centering \textbf{Регулярное выражение}\end{minipage} &
        \begin{minipage}[t]{4cm}\centering \textbf{Входная цепочка}\end{minipage} &
        \begin{minipage}[t]{4cm}\centering \textbf{Ожидаемый результат}\end{minipage} &
        \begin{minipage}[t]{4cm}\centering \textbf{Результат}\end{minipage} \\ \hline
        a* & a & соответствует & соответствует \\ \hline
        a* & b & не соответствует & не соответствует \\ \hline
        a* & aaaa & соответствует & соответствует \\ \hline
        a* & пустая & соответствует & соответствует \\ \hline

        (ab*)+ab & aab & соответствует & соответствует \\ \hline
        (ab*)+ab & abab & соответствует & соответствует \\ \hline
        (ab*)+ab & aaab & соответствует & соответствует \\ \hline
        (ab*)+ab & ab & не соответствует & не соответствует \\ \hline
        (ab*)+ab & пустая & не соответствует & не соответствует \\ \hline
        
        a(ab*)* & a & соответствует & соответствует \\ \hline
        a(ab*)* & aab & соответствует & соответствует \\ \hline
        a(ab*)* & ab & не соответствует & не соответствует \\ \hline
        a(ab*)* & пустая & не соответствует & не соответствует \\ \hline
    \end{tabular}
\end{table}


\section{Результат работы программы}

Результаты работы программы для регулярного выражения \texttt{a(ab*)*} приведены на рисунках \ref{img:nfa.pdf}--\ref{img:mfa.pdf}.

\imgs{nfa.pdf}{h!}{0.7}{НКА для регулярного выражения}

\imgs{dfa.pdf}{h!}{0.65}{ДКА для регулярного выражения}

\imgs{mfa.pdf}{h!}{0.65}{Минимизированный ДКА алгоритмом Бржозовского}

\clearpage
\section{Контрольные вопросы}

\begin{enumerate}
  \item Какие из следующих множеств регулярны? Для тех, которые регулярны, напишите регулярные выражения.
  \begin{enumerate}
      \item Множество цепочек с равным числом нулей и единиц.
      
      \textbf{Ответ:} Не является регулярным множеством.

      \item Множество цепочек из \texttt{\{0, 1\}*} с четным числом нулей и нечетным числом единиц.
      
      \textbf{Ответ:} Является регулярным множеством. 

      \textbf{Пример:} ((0110)|(1001)|(1010)|(0101)|(11)|(00))*1
      
      ((0110)|(1001)|(1010)|(0101)|(11)|(00))*

      \item Множество цепочек из \texttt{\{0, 1\}*}, длины которых делятся на 3.
      
      \textbf{Ответ:} Является регулярным множеством. 

      \textbf{Пример:} ((0|1)(0|1)(0|1))*

      \item Множество цепочек из \texttt{\{0, 1\}*}, не содержащих подцепочки 101.
      
      \textbf{Ответ:} Является регулярным множеством. 

      \textbf{Пример:} ((0*00)|1)*
  \end{enumerate}
  \item Найдите праволинейные грамматики для тех множеств из вопроса 1, которые регулярны.
  \begin{enumerate}
      \item \begin{equation}
          \begin{split}
              S \to 0110S \\
              S \to 1001S \\
              S \to 1010S \\
              S \to 0101S \\
              S \to 11S \\
              S \to 00S \\
              S \to 1A \\
              A \to 0110A \\
              A \to 1001A \\
              A \to 1010A \\
              A \to 0101A \\
              A \to 11A \\
              A \to 00A \\
              A \to \epsilon
          \end{split}
      \end{equation}
      \item \begin{equation}
          \begin{split}
              S \to 0A \\
              S \to 1A \\
              S \to \epsilon \\
              A \to 0B \\
              A \to 1B \\
              B \to 0S \\
              B \to 1S
          \end{split}
      \end{equation}
      \item \begin{equation}
          \begin{split}
              S \to A \\
              S \to 1S \\
              S \to \epsilon \\
              A \to 0A \\
              A \to 00S
          \end{split}
      \end{equation}
  \end{enumerate}
  \item Найдите детерминированные и недетерминированные конечные автоматы для тех множеств из вопроса 1,
  которые регулярны.
  \begin{enumerate}
      \item \imgs{3a.pdf}{h!}{0.48}{ДКА для первого регулярного выражения}
      \clearpage
      \item \imgs{3b.pdf}{h!}{0.5}{ДКА для второго регулярного выражения}
      \item \imgs{3c.pdf}{h!}{0.5}{ДКА для третьего регулярного выражения}
  \end{enumerate}
  \item Найдите конечный автомат с минимальным числом состояний для языка, определяемого автоматом M = (\{A,
  B, C, D, E\}, \{0, 1\}, d, A, \{E, F\}), где функция в задается таблицей
  \imgs{question4.png}{h!}{0.3}{Таблица для 4 вопроса}

  \imgs{4-dfa.pdf}{h!}{0.67}{ДКА для языка, определяемого автоматом M}
  \imgs{4-min-dfa.pdf}{h!}{0.67}{Минимизированный ДКА для языка, определяемого автоматом M}
\end{enumerate}


\clearpage
\section{Код прогаммы}

В листингах \ref{lst:main}--\ref{lst:builder} представлен код программы.

\mylisting[python]{main.py}{}{Основной модуль программы}{main}{}
\mylisting[python]{automata.py}{}{Классы НКА и ДКА}{automata}{}
\mylisting[python]{builder.py}{}{Функции перевода из автомата в автомат}{builder}{}

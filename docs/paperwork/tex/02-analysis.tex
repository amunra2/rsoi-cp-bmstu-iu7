\chapter{Аналитическая часть}

\section{Разбор аналогов}

Для успешной разработки веб-сайта аренды книг в библиотеках города важно проанализировать существующие аналогичные решения. Это позволит выделить сильные и слабые стороны уже реализованных проектов и использовать полученные данные для создания функционального и удобного ресурса. Рассмотрим несколько популярных примеров аналогов.

\begin{enumerate}
  \item \textbf{ЛитРес: Библиотека} \cite{litres}
  Это один из крупнейших российских онлайн-сервисов, предоставляющий доступ к электронной библиотеке, включая аренду книг для библиотек. Он активно сотрудничает с государственными библиотеками, предлагая пользователям доступ к бесплатным и платным материалам.

  Плюсы:
  \begin{itemize}
    \item Широкий выбор книг. Пользователи могут арендовать книги из огромной коллекции, включая новинки.
    \item Удобный интерфейс. Сайт и мобильное приложение имеют интуитивно понятный дизайн и структуру.
    \item Синхронизация с библиотеками. Возможность брать книги в аренду через конкретные библиотеки.
    \item Многоформатность. Книги доступны как в электронном, так и в аудиоформатах.
  \end{itemize}

  Минусы:
  \begin{itemize}
    \item Ограниченный бесплатный доступ. Большая часть контента платная, что может ограничивать круг пользователей.
    \item Отсутствие персонализации. Недостаточно индивидуальных рекомендаций и настроек под пользователя.
    \item Нет работы с физическими книгами. Платформа ориентирована в основном на электронные ресурсы, что не всегда удовлетворяет потребности пользователей, предпочитающих бумажные издания.
  \end{itemize}

  \item \textbf{Московская электронная библиотека} \cite{moscowlib}
  Государственный проект, предоставляющий жителям Москвы бесплатный доступ к библиотечным фондам, включая электронные книги и периодику.

  Плюсы:
  \begin{itemize}
    \item Бесплатный доступ. Пользователи могут бесплатно брать в аренду книги, что делает услугу доступной для всех.
    \item Интеграция с городскими библиотеками. Прямое взаимодействие с реальными библиотеками города, что упрощает получение информации о наличии книг.
    \item Широкий выбор. Библиотека содержит множество произведений различной тематики, включая учебную литературу.
  \end{itemize}

  Минусы:
  \begin{itemize}
    \item Устаревший интерфейс. Сайт выглядит морально устаревшим, что затрудняет навигацию и снижает пользовательский опыт.
    \item Малый функционал. Нет удобных инструментов для личного кабинета, отсутствуют продвинутые функции поиска и рекомендации.
    \item Отсутствие интеграции с физическими книгами. Хотя проект предоставляет электронные ресурсы, нет механизма аренды бумажных книг.
  \end{itemize}

  \item \textbf{OverDrive (международный аналог)} \cite{overdrive}
  Это популярная платформа для аренды книг, используемая библиотеками по всему миру. Она позволяет пользователям брать в аренду как электронные, так и аудиокниги через местные библиотеки.

  Плюсы:
  \begin{itemize}
    \item Интернациональность. Поддержка множества языков и сотрудничество с библиотеками по всему миру.
    \item Широкий ассортимент книг. Включает не только художественную литературу, но и учебные и исследовательские работы.
    \item Интеграция с мобильными устройствами. Поддержка мобильных приложений для различных платформ, что упрощает доступ к арендованным материалам.
  \end{itemize}

  Минусы:
  \begin{itemize}
    \item Ограниченное количество копий. Каждая библиотека ограничена количеством цифровых копий, что может привести к дефициту книг.
    \item Требование регистрации в местных библиотеках. Пользователю нужно быть зарегистрированным в библиотеке, чтобы получить доступ к ресурсам.
    \item Проблемы с доступом к физическим книгам. Платформа ориентирована на электронные и аудиокниги, а не на физические издания.
  \end{itemize}
\end{enumerate}

Изучив плюсы и минусы существующих решений, можно выделить несколько ключевых моментов для создания более эффективного веб-сайта аренды книг в библиотеках города.
\begin{enumerate}
  \item Современный и удобный интерфейс. Нужно избегать устаревших интерфейсов и предложить интуитивно понятный и современный дизайн.
  \item Бесплатный доступ и интеграция с библиотеками. Необходимо предоставить пользователям возможность арендовать книги бесплатно через городские библиотеки.
\end{enumerate}

Эти элементы помогут создать ресурс, способный конкурировать с существующими решениями и удовлетворить требования современных пользователей.


\section{Описание системы}
Разрабатываемый портал должен представлять собой систему для аренды книг в библиотеках города. 

Если пользователь хочет оформить заказ, то ему нужно пройти регистрацию, указав следующую информацию: фамилия, имя, номер телефона, адрес электронной почты, пароль. Для неавторизованных пользователей доступен только просмотр общей информации сайта: списка библиотек и книг в них.


\section{Функциональные требования к системе с точки зрения пользователя}
Портал должен обеспечивать реализацию следующих функций.

\begin{enumerate}
    \item Система должна обеспечивать регистрацию и авторизацию пользователей с валидацией вводимых данных.
    \item Аутентификация пользователей.
    \item Разделение всех пользователей на 3 роли:
    \begin{itemize}
      \item неавторизованный пользователь (гость);
	    \item авторизированный пользователь (пользователь);
		  \item администратор.
	\end{itemize}

  \item Предоставление возможностей \textbf{гостю, пользователю, администратору} представленных в таблице \ref{tbl:user-func}.
\end{enumerate}

\newpage
\begin{longtable}{|p{0.5cm}|p{15.5cm}|}
	\caption{Функции пользователей}
	\label{tbl:user-func} \\
	\hline
	
	\begin{rotatebox}[origin=r]{90}
		{ \textbf{Гость}}
	\end{rotatebox}
	& 
	1. Просмотр списка библиотек (включая фильтрацию по городу); \newline
	2. Просмотр списка книг в выбранной библиотеке; \newline
	3. Регистрация в системе; \newline
	4. Авторизация в системе. \\
	\hline
	
	\begin{rotatebox}[origin=r]{90}
		{ \textbf{Пользователь}}
	\end{rotatebox} 
	& 
  1. Авторизация в системе; \newline
	2. Просмотр списка библиотек (включая фильтрацию по городу); \newline
  3. Просмотр списка книг в выбранной библиотеке; \newline
	4. Получение информации о данных текущего аккаунта; \newline
	5. Просмотр всех своих арендованных книг во всех библиотеках с фильтрацией по статусу аренды \newline
	6. Получение детальной информации по конкретной аренде на имя текущего пользователя; \newline
	7. Оформление аренды книги на имя авторизованного пользователя; \newline
	8. Отмена аренды (возврат) на имя авторизованного пользователя. \\
	\hline
	
	\begin{rotatebox}[origin=r]{90}
	{ \textbf{Администратор}}
	\end{rotatebox} 
	& 
  1. Функции пользователя; \newline
	2. Просмотр статистики по сайту. \\	
	\hline
\end{longtable}


\section{Входные данные}
Входные параметры системы представлены в таблице \ref{tbl:input}.

\begin{longtable}{|p{3cm}|p{13cm}|}
	\caption{Входные данные}
	\label{tbl:input} \\
	\hline
	
	\textbf{Сущность} & \textbf{Входные данные} \\
	\hline
	\endfirsthead
	
	\hline
	\textbf{Сущность} & \textbf{Входные данные} \\
	\hline
	\endhead
	
	\hline
	\multicolumn{2}{c}{\textit{Продолжение на следующей странице}}
	\endfoot
	\hline
	\endlastfoot
	
	Регистрация пользователя
	&
	1. \textit{фамилия} не более 256 символов; \newline
	2. \textit{имя} не более 256 символов; \newline
	3. \textit{логин} не более 256 символов; \newline
	4. \textit{пароль} не более 128 символов; \newline
	5. \textit{номер телефона} в формате (+7XXXXXXXXXX); \newline
	6. \textit{роль} администратор или пользователь; \newline
	7. \textit{электронная почта} в формате (*@*.*). \\
	\hline

  Аутентификация пользователя
	&
	1. \textit{логин} не более 256 символов; \newline
	2. \textit{пароль} не более 128 символов. \\
	\hline

  Аренда книги
  & 
	1. \textit{идентификатор книги}; \newline
	2. \textit{идентификатор библиотеки}; \newline
	3. \textit{дата возврата} в формате ДД/ММ/ГГГГ. \\
	\hline

  Возврат книги
  & 
	1. \textit{идентификатор аренды}; \newline
	2. \textit{состояние книги} прекрасное / хорошее / плохое; \newline
	3. \textit{дата возврата} в формате ДД/ММ/ГГГГ. \\
	\hline

  Фильтр и пагинация библиотек
  & 
	1. \textit{город} не более 256 символов; \newline
	2. \textit{номер страницы} не менее 1; \newline
	3. \textit{объектов на странице} не менее 1. \\
	\hline

  Фильтр и пагинация книг
  & 
	1. \textit{показать закончившиеся} да / нет; \newline
	2. \textit{номер страницы} не менее 1; \newline
	3. \textit{объектов на странице} не менее 1. \\
	\hline

  Фильтр и пагинация аренд
  & 
	1. \textit{статус аренды} в аренде / возвращена вовремя / возвращена после срока; \newline
	2. \textit{номер страницы} не менее 1; \newline
	3. \textit{объектов на странице} не менее 1. \\
	\hline
\end{longtable}


\section{Выходные параметры}
Выходными параметрами системы являются web-страницы. В зависимости от запроса и текущей роли пользователя  они содержат следующую информацию (таблица \ref{tbl:output-data}).

\begin{longtable}{|p{0.5cm}|p{15.5cm}|}
	\caption{Выходные параметры}
	\label{tbl:output-data} \\
	\hline
	
	\begin{rotatebox}[origin=r]{90}
		{\textbf{Гость}}
	\end{rotatebox} 
	& 
	1. Список библиотек: \newline
    • \textit{название}; \newline
    • \textit{адрес}; \newline
    • \textit{город}. \\
	\cline{2-2}
    &
  2. Список книг: \newline
    • \textit{название}; \newline
    • \textit{автор}; \newline
    • \textit{жанр}; \newline
    • \textit{состояние}; \newline
    • \textit{количество в библиотеке}. \\
	\cline{2-2}
    &
	3. О сайте: \newline
   • \textit{общая информация о сайте}; \newline
   • \textit{правила сайта}; \newline
   • \textit{контактная информация поддержки}. \\
	\hline
	
	\begin{rotatebox}[origin=r]{90}
		{\textbf{Пользователь}}
	\end{rotatebox} 
	& 
	1. Список библиотек: \newline
    • \textit{название}; \newline
    • \textit{адрес}; \newline
    • \textit{город}. \\
	\cline{2-2}
    &
  2. Список книг: \newline
    • \textit{название}; \newline
    • \textit{автор}; \newline
    • \textit{жанр}; \newline
    • \textit{состояние}; \newline
    • \textit{количество в библиотеке}. \\
	\cline{2-2}
    &
	3. О сайте: \newline
   • \textit{общая информация о сайте}; \newline
   • \textit{правила сайта}; \newline
   • \textit{контактная информация поддержки}. \\
    \cline{2-2}
    &
  4. Детальная информация о пользователе, вошедшем в систему; \newline
    • \textit{фамилия}; \newline
    • \textit{имя}; \newline
    • \textit{логин}; \newline
    • \textit{номер телефона}; \newline
    • \textit{рейтинг} число от 0 до 100, характеризующее количество книг, которое может взять пользователь; \newline
    • \textit{электронная почта}.\\
	\cline{2-2}
	&
	5. Cписок взятых в аренду книг пользователя, вошедшего в систему: \newline
	• \textit{библиотека, в которой арендована книга} в соответствии с пунктом 1; \newline
	• \textit{арендованная книга} в соответствии с пунктом 2; \newline
	• \textit{дата взятия книги в аренду}; \newline
	• \textit{дата возврата книги}; \newline
  • \textit{статус аренды}. \\
	
  \hline
	\begin{rotatebox}[origin=r]{90}
		{\textbf{Администратор}}
	\end{rotatebox} 
	& 
	1. Список библиотек: \newline
    • \textit{название}; \newline
    • \textit{адрес}; \newline
    • \textit{город}. \\
	\cline{2-2}
    &
  2. Список книг: \newline
    • \textit{название}; \newline
    • \textit{автор}; \newline
    • \textit{жанр}; \newline
    • \textit{состояние}; \newline
    • \textit{количество в библиотеке}. \\
	\cline{2-2}
    &
	3. О сайте: \newline
   • \textit{общая информация о сайте}; \newline
   • \textit{правила сайта}; \newline
   • \textit{контактная информация поддержки}. \\
    \cline{2-2}
    &
  4. Детальная информация о пользователе, вошедшем в систему; \newline
    • \textit{фамилия}; \newline
    • \textit{имя}; \newline
    • \textit{логин}; \newline
    • \textit{номер телефона}; \newline
    • \textit{рейтинг} число от 0 до 100, характеризующее количество книг, которое может взять пользователь; \newline
    • \textit{роль в системе}; \newline
    • \textit{электронная почта}.\\
	\cline{2-2}
	&
	5. Cписок взятых в аренду книг пользователя, вошедшего в систему: \newline
    • \textit{библиотека, в которой арендована книга} в соответствии с пунктом 1; \newline
    • \textit{арендованная книга} в соответствии с пунктом 2; \newline
    • \textit{дата взятия книги в аренду}; \newline
    • \textit{дата возврата книги}; \newline
    • \textit{статус аренды}. \\
	\cline{2-2}
  &
  6. Статистика по порталу, собранная через сервис статистики:
    • \textit{идентификатор}; \newline
    • \textit{метод запроса} GET/POST/PATCH/DELETE/OPTIONS; \newline
    • \textit{url запроса}; \newline
    • \textit{числовой статус выполнения запроса}; \newline
    • \textit{время выполнения запроса} в формате ДД/ММ/ГГГГ (ЧЧ:ММ). \\
	\hline
\end{longtable}


\section{Состав системы}

Система будет состоять из фронтенда и 9 подсистем:
\begin{itemize}
	\item сервис-координатор;
	\item сервис регистрации и авторизации;
  \item сервис библиотек;
  \item сервис рейтинга;
  \item сервис аренды;
  \item сервис статистики;
  \item сервис kafka;
  \item сервис consumer;
  \item сервис zookeeper.
\end{itemize}


\subsection{Фронтенд}

\textit{Фронтенд} -- принимает запросы от пользователя по протоколу HTTP и возвращает ответ в виде HTML страниц, файлов стилей и TypeScript.


\subsection{Сервис-координатор}

\textbf{Сервис-координатор} -- сервис, который отвечает за координацию запросов внутри системы. Все сервисы портала (кроме сервиса регистрации и авторизации) должны взаимодействовать друг с другом через сервис-координатор, запросы с фронтенда в том числе сначала должны приходить на сервис-координатор, а затем перенаправляться на нужный сервис. При этом сервис-координатор отвечает за следующие действия.
\begin{enumerate}
	\item получения списка библиотек в городе с пагинацией от сервиса библиотек \ref{section:library};
	\item получения списка книг в библиотеке с пагинацией от сервиса библиотек \ref{section:library};
	\item получения списка книг, арендованных пользователем с пагинацией от сервисов библиотек \ref{section:library} и аренды \ref{section:reservation};
  \item получения рейтига пользователя от сервиса рейтинга \ref{section:rating};
  \item оформление аренды книги через сервисы библиотек \ref{section:library} и аренды \ref{section:reservation} с учетом рейтинга пользователя из сервиса рейтинга \ref{section:rating} (книг в аренде может быть не больше, чем рейтинг пользователя);
	\item возврат книги через сервисы библиотек \ref{section:library} и аренды \ref{section:reservation} и с изменением данных в сервисе рейтинга \ref{section:rating} (если книга возвращена в более плохом состояни и/или позже заявленного срока возврата, то за каждое снимается 10 рейтинга, иначе прибавляется 1 рейтинг).
\end{enumerate}


\subsection{Сервис регистрации и авторизации}

\textbf{Сервис регистрации и авторизации} отвечает за следующие действия.
\begin{enumerate}
	\item Регистрацию нового пользователя;
	\item Аутентификацию пользователя;
	\item Авторизацию пользователя;
  \item Получение данных пользователей;
  \item Изменение данных о пользователе;
	\item Удаление пользователя.
\end{enumerate}

Взаимодействие сервиса регистрации и авторизации с остальными сервисами должно осуществляться по протоколу OpenID Connect. Сам сервис представляет из себя Identity Provider \cite{idprovider}. Сервис регистрации и авторизации в своей работе используют базу данных, которая хранит следующую информацию:
\begin{itemize}
    \item Пользователь:
    \begin{itemize}
        \item \textit{уникальный идентификатор};
        \item \textit{логин};
        \item \textit{имя};
        \item \textit{фамилия};
        \item \textit{захешированный пароль};
        \item \textit{номер телефона};
        \item \textit{электронная почта};
        \item \textit{роль}.
    \end{itemize}
\end{itemize}


\subsection{Сервис библиотек} \label{section:library}

\textbf{Сервис библиотек} реализует следующие функции.
\begin{enumerate}
	\item Получение списка всех библиотек с фильтрацией и пагинацией;
	\item Получение информации о конкретной библиотеке;
	\item Создание библиотеки;
	\item Изменение библиотеки;
	\item Удаление библиотеки;
  \item Получение списка всех книг с фильтрацией и пагинацией;
	\item Получение информации о конкретной книге;
	\item Создание книги;
	\item Изменение книги;
	\item Удаление книги;
  \item Получение списка всех связей книг и библиотек с фильтрацией и пагинацией;
	\item Получение информации о конкретной связи библиотеки и книги;
	\item Создание связи библиотеки и книги
	\item Изменение связи библиотеки и книги
	\item Удаление связи библиотеки и книги.
\end{enumerate}

Сервис использует в своей работе базу данных:
\begin{itemize}
  \item Библиотека:
  \begin{itemize}
    \item \textit{уникальный идентификатор};
    \item \textit{название};
    \item \textit{город};
    \item \textit{адрес}.
  \end{itemize}

  \item Книга:
  \begin{itemize}
    \item \textit{уникальный идентификатор};
    \item \textit{название};
    \item \textit{автор};
    \item \textit{жанр};
    \item \textit{состояние}.
  \end{itemize}

  \item Связь библиотеки и книги:
  \begin{itemize}
    \item \textit{уникальный идентификатор};
    \item \textit{уникальный идентификатор библиотеки};
    \item \textit{уникальный идентификатор книги};
    \item \textit{количестов} книг в библиотеке по связии.
  \end{itemize}
\end{itemize}


\subsection{Сервис рейтинга} \label{section:rating}

\textbf{Сервис рейтинга} реализует следующие функции.
\begin{enumerate}
	\item Получение списка рейтингов всех пользователей с фильтрацией и пагинацией;
	\item Получение информации о конкретной рейтинге;
	\item Создание рейтинга;
	\item Изменение рейтинга;
	\item Удаление рейтинга.
\end{enumerate}

Сервис использует в своей работе базу данных:
\begin{itemize}
  \item Рейтинг:
  \begin{itemize}
    \item \textit{уникальный идентификатор};
    \item \textit{логин пользователя};
    \item \textit{рейтинг} число от 0 до 100.
  \end{itemize}
\end{itemize}


\subsection{Сервис аренды} \label{section:reservation}

\textbf{Сервис аренды} реализует следующие функции.
\begin{enumerate}
	\item Получение списка аренд всех пользователей с фильтрацией и пагинацией;
	\item Получение информации о конкретной аренде;
	\item Создание аренды;
	\item Изменение аренды;
	\item Удаление аренды.
\end{enumerate}

Сервис использует в своей работе базу данных:
\begin{itemize}
  \item Аренда:
  \begin{itemize}
    \item \textit{уникальный идентификатор};
    \item \textit{логин пользователя};
    \item \textit{уникальный идентификатор библиотеки};
    \item \textit{уникальный идентификатор книги};
    \item \textit{статус} в аренде / возвращена в срок / возвращена после срока;
    \item \textit{дата взятия книги в аренду};
    \item \textit{дата возврата книги}.
  \end{itemize}
\end{itemize}


\subsection{Сервис статистики}

\textbf{Сервис статистики} -- сервис, который отвечает за запись событий сервиса координатора в базу данных для осуществления возможности быстрого обнаружения, локализации и воспроизведения ошибки в случае её возникновения. Дает возможность получить статистику с пагинацией.

Сервис использует в своей работе базу данных:
\begin{itemize}
  \item Статистика:
  \begin{itemize}
    \item \textit{уникальный идентификатор};
    \item \textit{метод запроса} GET/POST/PATCH/DELETE/OPTIONS;
    \item \textit{url запроса};
    \item \textit{числовой статус выполнения запроса};
    \item \textit{время выполнения запроса}.
  \end{itemize}
\end{itemize}


\subsection{Сервис kafka}

\textbf{Сервис kafka} \cite{kafka} -- сервис, который необходим для сервиса статистики для сбора и обработки данных в реальном времени, что позволяет анализировать пользовательскую активность. Kafka поддерживает высокие объёмы данных и легко масштабируется, обеспечивая надёжную работу даже при значительных нагрузках. Благодаря встроенной отказоустойчивости и гарантии доставки сообщений, система статистики не потеряет важные данные при сбоях.


\subsection{Сервис consumer}

\textbf{Сервис consumer} -- сервис, который нужен kafka для получения, обработки и анализа данных, поступающих от producer в реальном времени. Kafka действует как посредник, обеспечивая доставку сообщений между различными сервисами, что позволяет consumer-серверам асинхронно получать данные и обрабатывать их по мере поступления. Это важно для поддержания высокой производительности и отказоустойчивости, так как Kafka распределяет нагрузку между несколькими consumer-серверами, помогая избежать перегрузки. Также Kafka гарантирует надёжную доставку сообщений, что позволяет consumer корректно обрабатывать каждое сообщение без риска потери данных. Наконец, она обеспечивает возможность параллельной обработки данных, что ускоряет анализ больших объёмов информации.


\subsection{Сервис consumer}

\textbf{Сервис consumer} -- сервис, который нужен kafka для получения, обработки и анализа данных, поступающих от producer в реальном времени. Kafka действует как посредник, обеспечивая доставку сообщений между различными сервисами, что позволяет consumer-серверам асинхронно получать данные и обрабатывать их по мере поступления. Это важно для поддержания высокой производительности и отказоустойчивости, так как Kafka распределяет нагрузку между несколькими consumer-серверами, помогая избежать перегрузки. Также Kafka гарантирует надёжную доставку сообщений, что позволяет consumer корректно обрабатывать каждое сообщение без риска потери данных. Наконец, она обеспечивает возможность параллельной обработки данных, что ускоряет анализ больших объёмов информации.


\subsection{Сервис zookeeper}

\textbf{Сервис zookeeper} \cite{zookeeper} -- сервис, который нужен kafka для управления и координации различных компонентов в своей распределённой системе. Вот ключевые задачи, которые решает Zookeeper в Kafka.

\begin{enumerate}
  \item Координация кластеров: Zookeeper помогает координировать работу брокеров (серверов Kafka) внутри кластера, отслеживая их состояние. Он сообщает Kafka о том, какие узлы доступны и активно работают, обеспечивая бесперебойное взаимодействие между ними.
  \item Управление метаданными: Zookeeper хранит важную информацию о топиках, партициях и распределении лидеров партиций среди брокеров. Это нужно для того, чтобы потребители (consumers) и производители (producers) могли эффективно взаимодействовать с нужными данными в кластере.
  \item Обнаружение лидера: Zookeeper определяет лидера для каждой партиции Kafka, который отвечает за запись и чтение данных. В случае сбоя одного из брокеров Zookeeper автоматически выбирает нового лидера для партиции, чтобы поддерживать непрерывную работу.
  \item Отказоустойчивость: Zookeeper обеспечивает высокую доступность и надёжность кластера Kafka, помогая восстанавливать компоненты после сбоев и поддерживать согласованное состояние всех узлов системы.
  Управление доступом: Zookeeper управляет доступом клиентов к Kafka и координирует изменения конфигурации, обеспечивая стабильность и безопасность работы кластера.
  \item Таким образом, Zookeeper является критически важным компонентом для обеспечения координации, отказоустойчивости и управления Kafka-кластером.
\end{enumerate}


\section{Требования к программной реализации}
\begin{enumerate}
  \item Требуется использовать СОА (сервис-ориентированную архитектуру) для реализации системы.
	\item Система состоит из микросервисов. Каждый микросервис отвечает за свою область логики работы приложения и должны быть запущены изолированно друг от друга.
	\item При необходимости, каждый сервис имеет своё собственное хранилище,  запросы между базами запрещены.
	\item При разработке базы данных необходимо учитывать, что доступ к ней должен осуществляться по протоколу TCP.
  \item Необходимо  реализовать  один  web-интерфейс  для  фронтенда.  Интерфейс  должен  быть  доступен  через  тонкий  клиент (браузер).
  \item Для межсервисного взаимодействия использовать HTTP (придерживаться RESTful).
  \item Выделить Gateway Service как единую точку входа и межсервисной коммуникации. В системе не должно осуществляться горизонтальных запросов.
	\item Необходимо предусмотреть авторизацию пользователей через интерфейс приложения.
	\item Код хранить на Github, для сборки использовать Github Actions.
	\item Каждый сервис должен быть завернут в docker.
\end{enumerate}


\section{Функциональные требования к подсистемам}

\textbf{Фронтенд} -- серверное  приложение, предоставляет пользовательский интерфейс и внешний API системы, при  разработке которого нужно учитывать следующее:
\begin{itemize}
  \item должен  принимать  запросы  по  протоколу  HTTP и формировать ответы пользователям в формате HTML;
	\item в зависимости от типа запроса должен отправлять последовательные запросы в соответствующие микросервисы;
  \item запросы к микросервисам необходимо осуществлять по протоколу HTTP;
  \item данные необходимо передавать в формате JSON;
  \item целесообразно использовать Tailwind для упрощения написания стилей.
\end{itemize}

\textbf{Сервис-координатор} -- это серверное приложение, которое должно отвечать следующим требованиям по разработке:
  \begin{itemize}
    \item обрабатывать запросы в соответствии со своим назначением, описанным в топологии системы;
    \item принимать и возвращать данные в формате JSON по протоколу HTTP;
    \item использовать очередь для отложенной обработки запросов (например, при временном отказе одного из сервисов);
    \item осуществлять деградацию функциональности в случае отказа некритического сервиса (зависит от семантики запроса);
    \item уведомлять сервис статистики о событиях в системе.
  \end{itemize}

 \textbf{Сервис регистрации и авторизации, сервис библиотек, сервис рейтинга, сервис аренды, сервис статистики} -- это серверные приложения, которые должны отвечать следующим требованиям по разработке:
  \begin{itemize}
    \item обрабатывать запросы в соответствии со своим назначением, описанным в топологии системы;
    \item принимать и возвращать данные в формате JSON по протоколу HTTP;
    \item осуществлять доступ к СУБД по протоколу TCP.
  \end{itemize}

\textbf{Сервис kafka, сервис consumer, сервис zookeeper} -- это серверное приложение, которое должно отвечать следующим требованиям по разработке:
  \begin{itemize}
    \item обрабатывать запросы в соответствии со своим назначением, описанным в топологии системы;
    \item принимать и возвращать данные в формате JSON по протоколу HTTP.
  \end{itemize}


\section{Пользовательский интерфейс}
Для реализации пользовательского интерфейса должен быть использован подход MVC (Model-View-Controller). Этот подход к проектированию интерфейса является популярным шаблоном проектирования, который помогает разделить логику приложения на три основных компонента: Модель (Model), Представление (View) и Контроллер (Controller). Этот подход позволяет улучшить структуру приложения, облегчить его тестирование и управление, а также разработка фронтенда и бекенда могут быть полностью разделены между собой, то есть можно вести независимую разработку.

Пользовательский интерфейс в разрабатываемой системе должен обладать следующими характеристиками:
\begin{itemize}
  \item Кроссбраузерность -- способность интерфейса работать практически в любом браузере любой версии. 
  \item <<Плоский» дизайн>> -- дизайн, в основе которого лежит идея отказа от объемных элементов (теней элементов, объемных кнопок и т.д.) и замены их плоскими аналогами.
  \item Расширяемость -- возможность легко расширять и модифицировать пользовательский интерфейс.
\end{itemize}
  
  
\section{Сценарий взаимодействия с приложением}
Приведем пример работы портала на примере выполнения запроса от пользователя на получение списка аренд пользователя.

\begin{enumerate}
  \item На фронтенд приходит запрос пользователя.
  \item Если пользователь был авторизован, то происходит получение токена аторизации. Затем выполняется запрос к сервису-координатору. Если данные корректны (данные поступили в ожидаемом формате) и проверка JWT-токена (проверка того, что токен был подписан известным серверу ключом и того, что срок действия токена ещё не истёк) (если он истёк или оказался некорректным, пользователю возвращается ошибка), то отправляется запрос на сервисы аренды и библиотек аренд. 
  \item Выполняются запросы к соответствующим эндпоинтам сервиса библиотек для получения даннных о библиотеках и книгах, осуществляется проверка корректности полученных данных (данные поступили в ожидаемом формате) и проверка JWT-токена (проверка того, что токен был подписан известным серверу ключом и того, что срок действия токена ещё не истёк). Если он истёк или оказался некорректным, пользователю возвращается ошибка. При успешной проверке токена сервис возвращает списки библиотек и книг сервису-координатору, который агрегирует полученные данные в одну таблицу и происходит возврат результата на фронтенд.
  \item Если ошибки нигде не произошло, то производится генерация HTML содержимого страницы ответа пользователю с использованием данных, полученных от сервиса-координатора. В ином случае генерируется страница с описанием ошибки.
\end{enumerate}
